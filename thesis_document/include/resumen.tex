\chapter*{Resumen}
\addcontentsline{toc}{chapter}{Resumen}

Este trabajo postula un y propone una metodología para generar un modelo que sea capaz de establecer y estimar los beneficios económicos asociados al uso del \emph{Building Information Modeling} (BIM) en los proyectos de construcción industrial. 

En particular, el método propuesto busca relacionar el nivel de integración BIM alcanzada por los proyectos y la manera en que esta impacta en el crecimiento de los costos de construcción, proponiendo, de esta manera, una métrica que permite modelar la relación inversa entre dicho nivel de integración (madurez BIM) y el crecimiento de los costos de construcción.

Dado que para generar el modelo que permite realizar la estimación de los sobrecostos según el nivel de madurez BIM se necesita tranformar la data de un proyecto de construcción en data útil para poder generar la relación que modele dicho comportamiento, este trabajo entrega herramientas para hacer la transformación de los datos que se desea analizar y para estimar los parámetros de interés. En particular, dichas herramientas fueron desarrolladas en el lenguaje de programción multipropósito Python.


