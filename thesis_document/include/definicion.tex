\chapter{Definición del Problema}

\section{Contexto}

Al momento de realizar esta Memoria, CODELCO no contaba con una herramienta que le permitiera hacer estimaciones sobre cuál es el real impacto de usar la metodología BIM.

A la fecha, lo que se conocía de la metodología en BIM en términos de los beneficios asociados a la reducción de costos eran los plantaeados en la \textit{National BIM survey} por \citeA{trejo2018estudio}. Allí, el autor indcaba que el uso del BIM impactaría reduciendo en un 33\% los costos iniciales de construcción y, en suma, de todo el ciclo de vida de un proyecto de construcción.

Esta aseveración, sin embargo, asume la condición ideal en que el BIM es aplicado con su máximo de madurez. Por lo que esta reducción de un 33\% bien puede tomarse como una condición de borde: un horizonte que todos los proyectos de construcción buscan alcanzar a la hora de aventurarse con la metodología BIM.


\subsection{Problema}

Así, pues, la principal motivación de CODELCO era conocer, en términos cuantitativos, el crecimientos de los costos de un proyecto de construcción de acuerdo al nivel de madurez BIM alcanzado por dicho proyecto respecto de su estimación inicial.