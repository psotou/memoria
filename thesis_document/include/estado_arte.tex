\chapter{Estado del Arte}

\section{La metodología BIM}

%Thus, \shortcite{lu2014cost} 
Ejemplos con \textbackslash \texttt{cite}: \shortciteA{lu2014cost}; \cite{lu2014cost}; \citeA{lu2014cost} y \citeNP{lu2014cost}.


%%%%%%%%%%%%%%%%%%%%%%%%%%%%%%%%%%%%%%%%%%%%%%%%%%%%%%%%%%%
\subsection{Ventajas de la metodología BIM}

La metodología BIM, considerada como plataforma de coordinación, puede tener dos efectos en el comportamiento de un contrato \cite{chang2014economic}: 
\begin{enumerate}
    \item Digitalizar el diseño en un conjunto de objetos paramétricos 3D podría reducir la incidencia de malas interpretaciones de la información del diseño provenientes de errores humanos durante la transferencia de información. Digitalizar reduce, de manera inherente, la incidencia en las órdenes de cambio, lo que permite que el dueño del proyecto esté menos expuesto a retrasos en los pagos.
    \item El uso de la metodología BIM podría hacer posible que las empresas subcontratistas incorporasen sus inputs en el diseño digital durante una etapa temprana. Además, el uso de BIM facilita la detección de interferencias que podrían resultar en un cambio durante la etapa de construcción. Todo lo anterior también impacta en la reducción de la exposición a cambios indeseados del dueño.
\end{enumerate}


Mainly a test to see if this shit is working properly